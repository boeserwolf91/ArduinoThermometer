\documentclass[a4paper]{scrreprt}

\usepackage[utf8]{inputenc} % Paket, dass angibt, dass der UTF-8 Zeichensatz verwendet wird.
\usepackage[ngerman]{babel}	% Paket für die neue deutsche Rechtschreibung
\usepackage{titlesec, blindtext, color}

\usepackage{graphicx}
\usepackage{listings}
\usepackage{hyperref}       % Paket, dass Links in die PDF einfügt.

\titleformat{\chapter}[hang]{\LARGE\bfseries}{\thechapter{.}\hspace{10pt}}{0pt}{\LARGE\bfseries} 
% Formatierung des \chapter befehls geaendert!


\author
{
	Stefan Wolf \\ 
	Angewandte Informatik - Mobile Systeme 2011 \\ 
	stefan.wolf@student.inf.hs-anhalt.de \\ 
	Matrikel-Nr: 4054391
}
\date{letzte Bearbeitung: \today}
\title
{
	Projekt "Arduino" \\ 
	\vspace{5 mm}
	{
		\large Erstellung eines Thermometers \\
	}
}

\hypersetup  % Setup für hyperred-Paket Einstellungen, können auch innerhalb \usepackage mit [] geschrieben werden
{
	%hidelinks = false,     		% hides links completaly
	bookmarks=true,         	% show bookmarks bar?
    unicode=false,          	% non-Latin characters in Acrobat’s bookmarks
    pdftoolbar=true,        	% show Acrobat’s toolbar?
    pdfmenubar=true,        	% show Acrobat’s menu?
    pdffitwindow=false,     	% window fit to page when opened
    pdfstartview={FitH},    	% fits the width of the page to the window
    pdftitle={Arduino},    	% title
    pdfauthor={Stefan Wolf}, 	% author
    pdfsubject={Erstellung eines Thermometers},   % subject of the document
    pdfcreator={Stefan Wolf},   % creator of the document
    pdfproducer={Stefan Wolf}, 	% producer of the document
    pdfkeywords={Arduino} {UNO} {Display} {Thermometer}, % list of keywords
    pdfnewwindow=true,      	% links in new window
    colorlinks=true,       		% false: boxed links; true: colored links
    linkcolor=black,          	% color of internal links (change box color with linkbordercolor)
    citecolor=black,        	% color of links to bibliography
    filecolor=magenta,      	% color of file links
    urlcolor=cyan           	% color of external links
}

\begin{document}
	\maketitle			% Erstellt eine Titelseite. Beinhaltet Autor, Datum, Titel
		
	\tableofcontents	% Erstellt ein Inhaltsverzeichnis
	\setcounter{tocdepth}{3}	% Gibt die max. Gliederungstiefe an, die im Inhaltsverzeichnis gezeigt wird
	\chapter{Einleitung und Motivation}	
		\section{Entstehung der Projektidee}
			Als ich zum ersten Mal in die Arduinoveranstaltung meiner Hochschule gekommen bin, 
			hatte ich keine Ahnung,	worum es sich dabei "uberhaupt genau handelt.
			Daher war es klar, dass ich mich erstmal mit den Basics vertraut machen muss.
			Auf Grund dessen habe ich mir zuerst ein paar B"ucher besorgt 
			und in diesen gest"obert. \\
			Allerdings ging die ganze Problematik nach kurzer Zeit schon relativ leicht von der Hand.   
			Deshalb musste jetzt langsam eine Idee her. F"ur mich stand fest, 
			dass das ganze nichts mit LED-W"urfeln oder "Ahnliches zutun haben sollte. \\
			Nach langem Hin und Her hab ich mir gedacht, dass ich mir ja ein Display 
			besorgen k"onnte und auf diesem irgendetwas anzeige.
			Das war eigentlich die ganze Entstehung der Idee. Da mir nichts besseres eingefallen
			ist, wollte ich eine Temperatur auf dem Display anzeigen lassen.
			
		\section{Vorstellung der Idee}	
			Bei dem Projekt geht es darum eine Temperatur auf dem Display anzeigen zu lassen.
			Zwischenzeitlich hatte sich die Idee eins, zwei mal ver"andert. So sollte mal 
			zus"atzlich	zur aktuellen Temperatur noch der minimale und der maximale 
			gemessene Wert angezeigt werden. Die Idee wurde allerdings nach den ersten
			Entw"urfen der Anzeige wieder verworfen. Auch der Plan die Temperatur 
			ebenfalls in Kelvin und Grad Fahrenheit anzeigen lassen zu können, 
			wurde wieder abgewendet.\\
			\newline
			Meine Hauptintention lag in der Ideesammelphase darin, das genaue 
			Zusammenwirken der einzelnen Bauteile und deren Ansteuerung 
			kennenzulernen. Auch das Arbeiten mit den Datenbl"attern
			musste trainiert werden. \\
			Zu diesem Zeitpunkt dachte ich noch, dass es sich bei diesem Projekt
			um eine leicht zu l"osende Aufgabe handelt. Diese hat sich aber schnell schon
			als schwieriger herausgestellt als vorher gedacht. Aber dazu an sp"aterer Stelle mehr.   
			
	\chapter{Konzept}
		%wie soll es gemacht werden, welche Mittel etc.
		%	-Vorstellung der Bauteile mit Bildern etc.
		In der Konzeptionsphase habe ich mich damit befasst, alle ben"otigten Arbeitsmittel 
		f"ur das Projekt zu besorgen.
		Im folgenden werde ich meine Teile auflisten und deren Funktion erl"autern. 
		Bilder zu den einzelnen	Teilen befinden sich im Anhang.
		
		\begin{itemize}
			\item \textbf{Arduino-StarterKit} \\
				Das Arduino Starterkit haben wir im Rahmen der Veranstaltung bekommen.
				Aus diesem Starterkit ben"otigte ich folgende Teile:
				\begin{itemize}					
					\item \textbf{Kabel} \\
					Viele Kabel aus diesem Kit wurden f"ur das Projekt verwendet. 
					Diese braucht man um die einzelnen Bauteile miteinander zu verbinden.
					
					\item \textbf{Potentiometer} \\
					Das Potentiometer aus diesem Kit wurde zur Regelung des Kontrastes 
					genommen. 
					
					\item \textbf{Temperatursensor LM35} \\
					Um die Temperatur der Umgebung zu messen und auf dem Display auszugeben,
					wird dieser Temperatursensor genommen.
					
					\item \textbf{Steckplatine} \\
					Da ich alle Teile aus dem Kit wieder abgeben muss, konnte ich an diese 
					keine Kabel anl"oten. 
					Zum Verbinden der einzelnen Teile wurde daher die Steckplatine verwendet.
				\end{itemize}
				
			\item \textbf{LCD-Display EA DIP204-4} \\
			Dabei handelt sich um meinen verwendeten Display. W"ahrend des Projektes wollte ich
			die Funktionsweise dieses Teils lernen. Auf Grund dessen sind auch Teile programmiert, 
			die nicht wirklich gebraucht werden. \\
			Beim Endprodukt ist dieses Ger"at daf"ur zust"andig, die aktuelle Temperatur 
			anzuzeigen. \\
			Das Display besitzt 18 Anschl"u"se. Es hat 2 Seiten mit je 9-St"uck mit einem Rasterma"s
			von 2,0mm. Da die Steckplatine des Arduinos im Rasterma"s von 2,54mm best"uckt ist, 
			ist es nicht m"oglich, dieses Display direkt anzuschlie"sen.
			\newpage
			\item \textbf{2x Buchsenleisten} \\
			Bei diesen Buchsenleisten handelt es sich um 2 10-polige, welche ein Rasterma"s von 2mm
			besitzen. Diese wurden an das Display gesteckt und daran Kabel angel"otet.
			Dieses Teil w"are nicht von N"oten gewesen. Allerdings wollte ich, da ich noch nie
			vorher etwas gel"otet hatte, das Display nicht kaputt machen.
			
			\item \textbf{Festspannungsregler} \\
			Dieses Teil regelt eine Spannung auf 3.5V.
			Der Festspannungsregler wurde daf"ur verwendet, die 5V des Arduinos f"ur die 
			Hintergrundbeleuchtung des Displays auf 3.5V zu regeln. 			 
		\end{itemize}
		
	\chapter{Umsetzung}
		Die Umsetzungsphase werde ich im Folgenden in die einzelnen Problem- bzw.
		Herausforderungsphasen unterteilen. Der Schwierigkeitsgrad der einzelnen Punkte 
		ist dabei allerdings nicht gleichm"a"sig verteilt. Das bedeutet, dass einige
		Unterpunkte leichter zu erledigen waren als Andere. \\
		Im Anhang befinden sich wieder Bilder aus den einzelnen Phasen. 
		Diese sind zus"atzlich entsprechend nummeriert.
		
		\section{L"oten}
			Der erste Schritt den es zu erledigen galt, war das Anl"oten von Kabeln an
			die Buchsenleisten. Da ich mich mit dieser Aufgabe noch nie besch"aftigt hatte,
			habe ich einen Kumpel um Hilfe gebeten.
			Dieser hat mir bei der Realisierung dieses Aufgabenteils geholfen. \\
			In der L"otphase wurde jeweils ein Kabelende eines Kabels mit einem Pin 
			der Buchsenleiste verbunden. Das Andere wurde verzinnt,	damit dieses sich 
			leichter mit der Steckplatine verbinden l"asst.
			Damit war das Problem abgeschlossen und ich konnte mich um den 
			n"achsten Punkt k"ummern.
		
		\section{Display Anschlie"sen}
			Beim Anschlie"sen der Pins war es erstmal notwendig, sich "uber die 
			einzelnen Funktionen schlau zu machen. Daf"ur musste ein Blick in 
			das Datenblatt des Displays geworfen werden. Im Folgenden werde ich die
			einzelnen Pins auflisten und erl"autern.
			
			\begin{itemize}
				\item \textbf{Pin 1} \\
				Dabei handelt es sich um den Ground bzw. Masse-Pin.
				
				\item \textbf{Pin 2} \\
				Dies ist der Pluspol des Displays. Dieser ben"otigt eine 5V 
				Stromversorgung.
				\newpage
				\item \textbf{Pin 3} \\
				Bei diesem Pin handelt es sich um die Kontrastspannung des Displays.
				Diese habe ich an einen Potentiometer angeschlossen, um damit den
				Kontrast auf einfache Art und Weise regeln zu k"onnen. 
				Allerdings habe ich bemerkt, dass das Regeln "uber ein Potentiometer
				wenig Sinn macht, da der Toleranzspannungbereich ziemlich klein ist und sich
				daher wenig einstellen l"asst.
				
				\item \textbf{Pin 4} \\
				Mithilfe dieses Pins l"asst sich dem Display sagen, ob Befehle oder Daten
				zum Display geschickt werden.
				Wenn Befehle auf den Display geschickt werden, ist dieser Pin LOW,
				bei Daten HIGH.
				
				\item \textbf{Pin 5} \\
				Der R/W-Pin teilt dem Display mit, ob im n"achsten Schritt auf das Display
				geschrieben oder vom Display gelesen wird. Daf"ur muss der Pin mit LOW belegt
				sein, wenn geschrieben werden soll und mit HIGH beim Lesen.
				
				\item \textbf{Pin 6} \\
				Bei diesem Pin handelt es sich um den Enable-Pin. Die Wirkungsweise 
				dieses Pins herauszufinden, hat mir die meiste Zeit des Projektes 
				gekostet. Das lag einfach daran, dass der Enable-Pin nur ganz kurz
				im Datenblatt erw"ahnt wurde und sozusagen vom Leser erwartet wurde,
				dass er das daf"ur notwendige Know-how besitzt. \\
				Dieser Pin muss nachdem alle Pins richtig gesetzt wurden einen Flankenwechsel
				von High auf Low haben. Nachdem dies passiert ist, f"uhrt das Display
				den gesendeten Befehl aus. \\
				Allerdings muss jetzt noch darauf geachtet werden, dass man, bevor das n"achste
				zum Display geschickt wird, wartet, bis der Befehl fertig ausgef"uhrt wurde.\\
				Daf"ur gibt es zwei M"oglichkeiten.
				Zum einen kann man die im Datenblatt stehenden Zeiten abwarten. Dies hat den 
				Vorteil, dass man den Prozess die n"otige Zeit schlafen legen k"onnte und die
				CPU diesen vorerst nicht ausf"uhren muss. Aber es gibt den Nachteil, dass man 
				unter Umst"anden zu lange wartet, da im Datenblatt die Maximaldauer 
				der einzelnen Befehle steht.
				Zum anderen k"onnte man ein Busy-Flag durch Polling abfragen. Der Vorteil hierbei
				besteht darin, dass man immer nur so lange wartet, wie n"otig ist und keine
				unn"utze Zeit verstreichen l"asst. Allerdings l"auft hierbei der Prozess immer 
				auf 100\% und die CPU kann sich ggf. in der Zeit nicht voll und ganz um andere 
				wichtige Prozesse k"ummern. \\
				Bei meinem Projekt hab ich mich für die erste Variante entschieden, 
				da ich keine zeitkritische Anwendung damit bedienen m"ochte.
				  
				\item \textbf{Pin 7 - Pin 14} \\
				Dabei handelt es sich um die 8 Datenpins von DB0 - DB7. Mit deren Hilfe, werden
				die Daten und Befehle spezifiziert.
				
				\item \textbf{Pin 15} \\
				Dieser Pin hat keine Funktion.
				
				\item \textbf{Pin 16} \\
				Dieser Pin ist der Reset-Pin. Mithilfe eines Flankenwechsels von High auf Low
				kann man damit das Display komplett zur"ucksetzen.
				
				\item \textbf{Pin 17} \\
				Hierbei handelt es sich um den Pluspol der LED-Hintergrundbeleuchtung. Dieser
				ist mit einem Festspannungsregler verbunden der die Spannung auf 3,5V 
				herunter regelt.
				
				\item \textbf{Pin 18} \\
				Dieser Pin ist der Minuspol der LED-Hintergrundbeleuchtung.
				 
			\end{itemize}
			Mithilfe dieser Informationen konnte man dann die Pins mit dem Arduino verbinden.
			Einen ausf"uhrlichen Schaltplan habe ich im Anhang hinterlegt. 
			Als das Display fertig angeschlossen war, habe ich mit der Ansteuerung begonnen.
			
		\section{Display Programmieren}	
			Nachdem das Anschlie"sen des Displays abgeschlossen war, habe ich damit angefangen
			mir zu "uberlegen, wie ich das Display jetzt testen kann. 
			Da ich keine Ahnung hatte, wie das Ganze genau funktioniert, habe ich damit begonnen
			eine kleine Displayklasse zu schreiben. Diese sollte anfangs nur ganz spartanische
			Methoden besitzen. So ist folgende kleine Headerdatei entstanden:
			
			\begin{lstlisting}[language=c++]
#define RW 12
#define RS 13

#define DB0 10
#define DB1 9
...
#define DB7 3

class Display
{
	private:
		/*
		 * Dieses Attribut beinhaltet die 8 Datenbits.
		 */
		int DB_ARRAY[8];
		/*
		 * Diese Methode bestimmt ob das Byte aus dem  
		 * value-Parameter 1 oder 0 ist und gibt dies 
		 * entsprechend zurueck.
		 */
		int getBit(int value, int bitNumber);
		
	public:
		/*
		 * Konstruktor hat DB_ARRAY initialisiert
		 */
		Display();
		/*
		 * Sollte Datenbyte an Display senden
		 */
		void writeData(int value);
		/*
		 * Sollte Befehl an Display senden
		 */
		void executeCommand(int value);
}
			\end{lstlisting} 
			Wie man sieht hatte der Enable-Pin zu diesem Zeitpunkt noch	gar keinen 
			Platz im Programm gefunden, da ich damit nicht umzugehen wusste.
			(Sieht man daran, dass die passende Pr"aprozessorvariable fehlt!) \\
			Nachdem diese Version nicht funktioniert hatte, hat es der Pin
			aber trotzdem ins Programm geschafft und nach ewigem Ausprobieren und 
			Ab"andern das Quelltextes, hat es dann auch irgendwann funktioniert. \\
			\newline
			Nachdem ich es hinbekommen hatte erfolgreich mit dem Display zu kommunizieren,
			musste ich es schaffen, float-Werte an den Display zu senden. Daher 
			mussten daf"ur auch noch Methoden geschrieben werden.
			Dieser Schritt war allerdings im Vergleich zum Vorherigen relativ leicht 
			und schnell erledigt. \\
			Als Letztes musste ich mir jetzt noch "uberlegen, wie ich die Werte
			auf dem Display darstelle. Ich habe mich daraufhin ziemlich schnell daf"ur 
			entschieden, die Temperatur "uber alle 4 Zeilen anzeigen zu lassen und mich
			mit Photoshop um ein paar Entw"urfe gek"ummert. Die Schwierigkeit dabei bestand
			darin, dass ich f"ur diese Aufgabe nur 8 eigene Zeichen definieren durfte.
			Nachdem ich 2 verschiedene Entw"urfe erstellt hatte, habe ich einen Kumpel nach
			seiner Meinung "uber diese gefragt. Nach kurzer Zeit hatte er mir ein Bild mit
			einer Schriftart gesendet, die er im Internet gefunden hatte. Diese ist dann 
			schlie"slich auch die finale Schriftart geworden. \\
			Als auch dieser Schritt erfolgreich abgearbeitet war, kam ich zum vorerst letzten
			Abschnitt meines Projektes. Ich musste jetzt schlie"slich noch den 
			Temperatursensor anbauen und programmieren.  

		\section{Temperatursensor Anschlie"sen und Programmieren}
			Da ich meine Grundkentnisse, w"ahrend ich die anderen Abschnitte abgearbeitet hatte,
			ausbaute, war dieser Schritt mit Abstand am einfachsten zu realisieren.
			Den Sensor musste man einfach nur "uber 3 Pins anschlie"sen. Zwei Anschl"usse 
			davon sind der Plus- und der Minuspol.
			Der dritte Pin ist die Ausgangsspannung aus dem Ger"at, die abh"angig von der 
			Temperatur ist. Dieser wird einfach nur an einen der analogen Eingang des 
			Arduinos angeschlossen. \\
			Um die aktuelle Temperatur auszulesen, muss man jetzt einfach nur den Pin auslesen
			und ein bisschen rechnen. Im folgenden zeige ich das Ganze im Quellcode:
			\begin{lstlisting}[language=c++]
...
/* 
 * adc - Analog/Digital-Converter 
 * gibt Wert von 0 bis 1023 (2^10) zurueck!
 */
int adcValue = analogRead(temperatureSensorPin);
/* milliVolt - Voltzahl in mV, die am Sensor anliegt */
float millivolts = (adcValue / 1024.0) * milliVolt;

/* Sensor-Output ist 10mV pro Grad Celsius */
float celsius = millivolts / 10; 
...
			\end{lstlisting}
			Allerdings musste ich feststellen, dass die Temperatur manchmal ziemlich 
			gro"se Ausrei"ser hat.
			Auf Grund dessen habe ich die Temperatur mittels Mittelwertbestimmung erhalten.
			Ich habe immer die letzten 14 gemessenen Werte gespeichert und den Mittelwert aus
			diesen Zahlen und der neu gemessenen Zahl bestimmt, um diesem Effekt entgegen 
			zu wirken. 
		
		\section{Coderefactoring}		
		Im letzten Schritt musste der Code noch h"ubsch gemacht werden. Da man jetzt die
		komplette Funktionsweise kennt, konnte man sich nun dar"uber Gedanken machen, wie man
		den Quellcode am besten aufbaut. \\
		Ich habe mich daf"ur entschieden, das ganze mit 3 Klassen und 2 Strukturen zu realisieren.
		Den kompletten Quellcode findet man ebenfalls komplett kommentiert im Anhang.		
		
		
	\chapter{Ergebnisse bzw. Fazit}
		W"ahrend des Projektes hab ich eine Menge positive wie negative Erfahrungen dazu 
		gewonnen und ich denke, dass sich das Ergebnis nicht zu verstecken braucht.
		Ich habe gelernt, wie ich mit elektronischen Bauteilen umzugehen habe, wie man 
		mit diesen kommunizieren kann und wie man ein solches Projekt aufbaut.
		Leider musste ich feststellen, dass die Informationen die in den einzelnen
		Datenbl"attern stehen nicht immer die Wahrheit erz"ahlen oder gar n"utzlich sind. 
		Beispielsweise war die Definition von eigenen Zeichen f"ur das Display 
		vollkommen falsch beschrieben. \\
		Au"serdem konnte ich feststellen, dass mir das Programmieren mehr Spa"s bereitet, als 
		das L"oten und Zusammenstecken. Nichtsdestotrotz hat es einen riesen Spa"s gemacht
		und ich w"urde kommenden Arduino-Projekten nicht von vornherein einen Stein in den
		Weg legen. \\
		Das bedeutet, dass ich auch in Zukunft daf"ur offen w"are, eine Problemstellung 
		dieses Typs zu realisieren.
		Interessant f"ande ich es, eine kleine Art Betriebssystem mit einfachem Dispatcher und
		einfachem Scheduler f"ur das Arduino zu bauen.
		Sodass man unter Umst"anden mehr als ein Projekt mit dem Arduino verwalten k"onnte.
		Vielleicht gibt es ja irgendwann die daf"ur notwendige Zeit.
		
\end{document}
